\chapter{A aplicação}

Até o momento, foram apresentados diversos aspectos relacionados à implementação da API. Neste capítulo, será exposto o resultado final do sistema. A seguir, são demonstradas as chamadas da API, evidenciando funcionalidades como cadastro de usuários, autenticação (login) e criação de tickets. Essa implementação assegura a rastreabilidade das atividades realizadas em cada \textit{ticket}, contribuindo significativamente para o controle e a gestão eficiente dos chamados no sistema.

\section{Criar Usuário}
A Figura \ref{fig:criarUsuario} apresenta a tela do postman referente a criação de usuário no sistema. Esse recurso permite que novos usuários sejam cadastrados com seus dados pessoais e credenciais de acesso. No teste realizado, a funcionalidade de criação de usuário foi executada com sucesso, validando que o sistema está apto a registrar novos perfis e armazená-los corretamente no banco de dados.

Durante o teste, foram inseridas informações válidas em todos os campos obrigatórios do formulário, e após a submissão, o sistema retornou uma confirmação positiva, indicando que o usuário foi criado com êxito.

\begin{figure}[H]
    \centering
    \includegraphics[scale=0.5]{figuras/sistema/teste de funcionamento/usuario/criar_usuario.png}
    \caption{Criar Usuário}
    \label{fig:criarUsuario}
\end{figure}

A Figura \ref{fig:erros_usuarios} apresenta dois tipos de erros que podem ocorrer durante o processo de criação de usuários no sistema. Na subfigura \ref{fig:pac1}, é exibida a mensagem de erro gerada quando o email de usuário informado já está cadastrado no sistema, impedindo a duplicidade de registros. Já a subfigura \ref{fig:pac2} mostra a mensagem apresentada quando o formulário é submetido com campos obrigatórios não preenchidos, o que reforça a necessidade de validação dos dados inseridos pelo usuário antes do envio.
\begin{figure}[h]
    \begin{subfigure}{0.48\linewidth}    \includegraphics[width=1\textwidth,height=8cm]{figuras/sistema/teste de funcionamento/usuario/criar_usuario_error1.png}
    \subcaption{Erro de usuário duplicado}
    \label{fig:pac1}
    \end{subfigure}
    \begin{subfigure}{0.48\linewidth}    \includegraphics[width=1\textwidth,height=8cm]{figuras/sistema/teste de funcionamento/usuario/criar_usuario_error2.png}
    \subcaption{Erro de campos ausentes}
    \label{fig:pac2}
    \end{subfigure}
\caption{Erros na criação de usuários} \label{fig:erros_usuarios}
\end{figure}

\newpage

\section{Realizar Login}
A Figura \ref{fig:realizarlogin} apresenta a tela do postman para realizar o login do usuário no sistema. Ao informar o e-mail e a senha previamente cadastrados, o sistema valida as credenciais, retornando um token de acesso (JWT) com validade de 1 hora, além de um refresh token com validade de 8 horas. Esses mecanismos garantem a segurança do acesso às funcionalidades do sistema, permitindo a renovação da sessão de forma transparente e controlada. O teste confirmou que o fluxo de autenticação está funcionando corretamente, proporcionando uma experiência segura e eficiente para o usuário.
\begin{figure}[H]
    \centering
    \includegraphics[scale=0.3]{figuras/sistema/teste de funcionamento/login/realizar_login.png}
    \caption{Realizar Login}
    \label{fig:realizarlogin}
\end{figure}

\newpage

\section{Criar \textit{Ticket}}
A Figura \ref{fig:criarTicket} apresenta a interface do postman para a funcionalidade de criação de \textit{tickets} no sistema. Essa funcionalidade é fundamental para registrar e gerenciar solicitações feitas pelos usuários. Durante o teste, foi preenchido o formulário com os dados obrigatórios para a criação do \textit{ticket}, como o título, que resume a solicitação, e a descrição, onde são detalhadas as informações do problema ou da demanda a ser atendida, além de informar também o departamento e categoria do chamado.

Após o preenchimento correto dos campos, o usuário confirmou a criação do \textit{ticket}, e o sistema respondeu com uma mensagem de sucesso, indicando que o registro foi armazenado com êxito no banco de dados. Esse teste validou que a API está processando corretamente os dados enviados, vinculando o \textit{ticket} ao usuário autenticado e disponibilizando-o para acompanhamento e resolução futura.
\begin{figure}[H]
    \centering
    \includegraphics[scale=0.4]{figuras/sistema/teste de funcionamento/ticket/criar_ticket.png}
    \caption{Criar \textit{Ticket}}
    \label{fig:criarTicket}
\end{figure}

\section{Registrar Atendimento}

A Figura \ref{fig:registraratendimento} apresenta a interface de registro de atendimento de um \textit{ticket}. Essa funcionalidade é essencial para documentar as ações realizadas durante o processo de suporte, permitindo maior controle sobre o histórico de cada solicitação. Ao iniciar um atendimento, o sistema registra automaticamente o horário de início, e, ao finalizá-lo, calcula o tempo total gasto com base na diferença entre os horários de início e término.

Durante o teste, um atendimento foi iniciado a partir de um \textit{ticket} previamente criado. Após a inserção de uma descrição com os detalhes das ações executadas, o atendimento foi finalizado com sucesso. O sistema então apresentou o tempo decorrido, confirmando que o cálculo de duração do atendimento está funcionando corretamente. Esse recurso é importante tanto para a avaliação da eficiência da equipe quanto para a geração de indicadores de desempenho e relatórios gerenciais.

\begin{figure}[H]
    \centering
    \includegraphics[scale=0.4]{figuras/sistema/teste de funcionamento/registrar atendimento/registrar_atendimento.png}
    \caption{Registrar atendimento de algum \textit{ticket}}
    \label{fig:registraratendimento}
\end{figure}

\section{Criar Comentário}

A Figura \ref{fig:criarComentario} exibe a funcionalidade de criação de comentários em um \textit{ticket}. Esse recurso tem como objetivo registrar observações, atualizações ou interações complementares ao atendimento principal, facilitando a comunicação entre os usuários envolvidos e mantendo um histórico detalhado do progresso da solicitação.

Durante o teste, foi inserido um comentário em um \textit{ticket} já existente. O sistema exigiu que o campo de texto fosse preenchido com uma mensagem válida, e após a submissão, confirmou o registro do comentário com sucesso. A funcionalidade demonstrou estar funcionando corretamente, associando o conteúdo ao \textit{ticket}.

\begin{figure}[H]
    \centering
    \includegraphics[scale=0.4]{figuras/sistema/teste de funcionamento/comentario/criar_comentario.png}
    \caption{Criar Comentário de um \textit{ticket}}
    \label{fig:criarComentario}
\end{figure}
