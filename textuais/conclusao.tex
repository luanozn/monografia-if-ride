\chapter*{CONCLUSÃO} 
\addcontentsline{toc}{chapter}{CONCLUSÃO}
\thispagestyle{plain}

Este trabalho teve como objetivo o desenvolvimento de uma API para gerenciamento de \textit{tickets} de atendimento, visando atender às necessidades de empresas que buscam melhorar seus processos internos de suporte e relacionamento com clientes. A solução proposta se destacou por sua flexibilidade, segurança e capacidade de personalização, oferecendo um sistema capaz de se adaptar às diferentes realidades organizacionais.

Ao longo do desenvolvimento, foram aplicados conceitos sólidos de engenharia de software, como a utilização de autenticação com JSON Web Token (JWT), arquitetura modular e boas práticas de segurança e escalabilidade. Além disso, o sistema foi estruturado para permitir integrações com outros serviços corporativos, oferecendo recursos como categorização de chamados, definição de níveis de prioridade, notificações automáticas e controle de fluxo de atendimento.

Os testes realizados demonstraram a eficácia da aplicação em cenários reais de uso, com funcionalidades completas de criação, atualização, listagem e exclusão de \textit{tickets}, bem como o registro de atividades relacionadas. A modelagem da base de dados, aliada ao uso de frameworks robustos, garantiu um desempenho satisfatório mesmo com grandes volumes de dados.

A Tabela~\ref{table:comparativo_tickets_final} apresenta um comparativo entre o sistema desenvolvido e três das principais ferramentas comerciais de gerenciamento de \textit{tickets} (Zendesk, Freshdesk e Zoho Desk), destacando-se aspectos como capacidade de personalização, integração com sistemas legados, controle de acesso e custo. Os dados demonstram que a solução criada oferece vantagens competitivas relevantes, especialmente para organizações que demandam maior flexibilidade e adaptação a contextos específicos.

\begin{table}[ht]
\centering
\resizebox{\textwidth}{!}{%
\begin{tabular}{|l|c|c|c|c|}
\hline
\textbf{Característica} & \textbf{Zendesk} & \textbf{Freshdesk} & \textbf{Zoho Desk} & \textbf{Sistema criado} \\ \hline
Interface amigável e intuitiva & Sim & Sim & Sim & Sim \\ \hline
Customização de fluxos e regras & Alta & Baixa & Média & Alta \\ \hline
Integração com sistemas legados & Limitada & Limitada & Limitada & Avançada \\ \hline
Automação de \textit{tickets} e respostas & Sim & Sim & Sim & Sim \\ \hline
Gerenciamento de SLA e métricas & Sim & Sim & Sim & Sim \\ \hline
Controle de acesso e permissões & Limitado & Limitado & Limitado & Avançado \\ \hline
Preço & Caro & Mais acessível & Mais acessível & Mais acessível \\ \hline
\end{tabular}
}
\caption{Comparativo de Características em Sistemas de Gerenciamento de \textit{Tickets} para o sistema criado}
\label{table:comparativo_tickets_final}
\end{table}

\newpage

Como trabalhos futuros, poderá ser realizada a implementação de novas funcionalidades no sistema, como um dashboard analítico com gráficos e indicadores que permitam a visualização em tempo real dos \textit{tickets} abertos, em andamento, resolvidos, tempo médio de atendimento, entre outros dados relevantes para a gestão de desempenho das equipes de suporte.
Também é proposta a integração com um chatbot e a criação de um módulo de feedback, permitindo que o cliente avalie o atendimento após o encerramento do \textit{ticket}, contribuindo para a melhoria contínua do serviço.

No entanto, este trabalho apresenta algumas limitações. Por se tratar de uma aplicação do tipo API, não foi possível realizar testes em um ambiente real de produção, uma vez que seria necessário disponibilizar a aplicação em um servidor para simular o uso contínuo. Além disso, a ausência de uma interface gráfica (front-end) dificultou a validação do sistema por usuários finais, restringindo os testes à utilização de ferramentas como o Postman ou o Swagger UI, o que limita a avaliação da usabilidade e da experiência do usuário. A criação de uma aplicação front-end integrada é, portanto, um passo essencial para uma avaliação mais completa da solução proposta.