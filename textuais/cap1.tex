\chapter{Fundamentos básicos} \label{fundamentos_basicos}

\section{Câmpus}

O \textit{IF Goiano - Câmpus Urutaí} é uma instituição de ensino localizada na zona rural, instalada em uma área de 512 hectares na região da Estrada de Ferro, no sudeste do Estado de Goiás \cite{ifgoiano2024}. Fundado a partir de uma rica trajetória histórica que remonta à criação da Fazenda Modelo de Criação em 1918 – um marco que possibilitou a formação de profissionais para o setor agropecuário –, o campus evoluiu de Escola Agrícola para Escola Agrotécnica e, posteriormente, para a sua atual configuração como Instituto Federal Goiano \cite{turn0search1}. Ao longo de sua história, o campus tem desempenhado um papel fundamental na democratização do acesso à educação técnica, profissional e superior, contribuindo para o desenvolvimento regional e para a formação de profissionais qualificados que atendem às demandas do mercado local \cite{turn0search2}. Além disso, o IF Goiano - Campus Urutaí destaca-se pela integração entre ensino, pesquisa e extensão, fortalecendo seu compromisso com a inovação e a transformação social, aspectos que se refletem na oferta de uma ampla gama de cursos – desde o ensino médio integrado até programas de pós-graduação \cite{turn0search6}. Essa diversidade de modalidades de ensino ressalta a importância do campus como polo de desenvolvimento educacional e social na região.


\section{Principais desafios}

Uma das principais dificuldades enfrentadas pelos estudantes e servidores do Instituto Federal Goiano, especialmente no Câmpus Urutaí, refere-se à sua localização geográfica em área rural, caracterizada pelo acesso limitado a opções de transporte público e pela distância significativa dos centros urbanos. A ausência de transporte público nessa região, aliada ao fato de que
as cidades que compõem o entorno do câmpus também são bastante pequenas, resulta na falta de
uma alternativa eficiente ao transporte privado. Infelizmente, o transporte privado geralmente
é excessivamente caro, propenso a atrasos e não confiável, o que acaba criando um cenário
de monopólio para os usuários e limitando suas opções de escolha. \cite{propostaTCC2025}.

A situação do transporte no câmpus apresenta-se como um desafio significativo para a comunidade acadêmica. Conforme mencionado anteriormente, os custos associados ao transporte são elevados, o que impacta diretamente estudantes que enfrentam condições financeiras adversas. Muitos desses alunos, diante da dificuldade de arcar com o deslocamento diário, optam por residir nas dependências da instituição para viabilizar a continuidade dos estudos.
Embora algumas prefeituras ofereçam auxílios municipais destinados ao transporte até o Instituto, e existam bolsas de assistência estudantil para aqueles que comprovam baixa renda per capita, tais medidas ainda se mostram insuficientes. O auxílio municipal, por si só, não é capaz de reduzir significativamente os custos do transporte privado, mantendo-o inacessível para uma parcela considerável dos alunos. Além disso, as bolsas disponibilizadas não contemplam todos os estudantes em situação de vulnerabilidade, sendo limitadas tanto em número quanto em valor.
Diante desse cenário, é evidente que a dificuldade de acesso afeta não apenas os estudantes das cidades que recebem algum tipo de auxílio, mas agrava-se ainda mais para aqueles provenientes de municípios não contemplados. Esse contexto contribui para o aumento da evasão escolar e dificulta o acesso ao ensino para alunos de baixa renda, comprometendo a inclusão e a permanência no Instituto.


\section{Mobilidade Compartilhada: Conceito e Relevância}
De acordo com \cite{bonaldo2021}, a mobilidade compartilhada refere-se à utilização coletiva e otimizada de meios de transporte, apoiada em tecnologias de informação que possibilitam a integração de diferentes modais. Ele propõe que a mudança do paradigma da propriedade individual para o compartilhamento de veículos pode revolucionar o planejamento urbano, reduzindo custos e otimizando a infraestrutura das cidades. Essa perspectiva é reforçada por \cite{willemann2024}, que enfatiza as inovações tecnológicas como fator determinante para a viabilização de sistemas colaborativos de transporte. A autora também argumenta que a convergência entre tecnologia e mobilidade cria oportunidades para cidades inteligentes, onde os dados em tempo real permitem a melhor distribuição dos recursos disponíveis. Em suma, ambos os autores apontam para a necessidade de repensar o transporte urbano de forma integrada e sustentável, destacando que a mobilidade compartilhada não se trata apenas de uma alternativa econômica, mas de uma estratégia de transformação social.

\subsection{Impactos Ambientais da Mobilidade Compartilhada}
O benefício ambiental da mobilidade compartilhada reside na otimização da taxa de ocupação dos veículos. Conforme \citeonline{moro2022}, a redução do volume de veículos particulares em circulação correlaciona-se diretamente com a diminuição das emissões de gases poluentes. Do ponto de vista técnico, sistemas de caronas permitem que a mesma demanda de deslocamento seja suprida com menor consumo energético por passageiro. Além da redução de emissões, \citeonline{willemann2024} aponta que o fluxo de tráfego mais eficiente mitiga o tempo de ociosidade de motores em congestionamentos, potencializando a economia de combustível e a redução de custos operacionais para os usuários.

\subsection{Incentivos à Adoção da Mobilidade Compartilhada}
Para que a mobilidade compartilhada se consolide, é fundamental o desenvolvimento de políticas públicas e estratégias de incentivo. \cite{bonaldo2021} destaca que medidas como subsídios, investimentos em infraestrutura (por exemplo, ciclovias e pontos de compartilhamento) e campanhas de conscientização são essenciais para superar barreiras culturais e econômicas. \cite{bonaldo2021} enfatiza que a integração entre os setores público e privado pode facilitar a implementação de sistemas interconectados, aumentando a confiabilidade e a adesão dos usuários. Essa integração, segundo o autor, não apenas reduz os custos individuais com transporte, mas também promove uma maior inclusão social, ao oferecer alternativas acessíveis a diferentes faixas da população.

\subsection{Tecnologia, Segurança e Outras Considerações Importantes}
Os desafios da mobilidade compartilhada não se restringem apenas à sua implementação operacional, mas também envolvem aspectos tecnológicos e de segurança. \cite{moro2022} discute a necessidade de investimentos contínuos em sistemas de monitoramento, interfaces intuitivas e protocolos de segurança robustos, os quais são cruciais para garantir a confiança dos usuários. Os autores argumentam que a interoperabilidade entre plataformas e a regulamentação adequada são determinantes para a expansão sustentável desse modelo. Outro ponto importante refere-se à adaptação da infraestrutura urbana, que deve acompanhar as novas demandas geradas pelos serviços de mobilidade compartilhada, proporcionando espaços seguros e bem planejados para veículos e pedestres.


\subsection{A Mobilidade Compartilhada no IF Goiano - Campus Urutaí}

No contexto do Câmpus Urutaí, a mobilidade compartilhada atua como uma solução para a carência de transporte público em áreas rurais. A utilização de caronas entre a comunidade acadêmica permite otimizar os deslocamentos, reduzindo o custo logístico para estudantes e servidores. Atualmente, o compartilhamento ocorre de forma fragmentada, muitas vezes dependendo de redes sociais externas ou acordos informais.

A implementação de uma plataforma dedicada visa formalizar e automatizar esse processo. Segundo as definições de \citeonline{bonaldo2021} e \citeonline{willemann2024}, o uso de tecnologias de informação é o que permite a integração eficiente desses modais. Para o câmpus, o desenvolvimento de um aplicativo de caronas resolve o problema do alto valor das tarifas privadas, permitindo a divisão de custos de forma auditável e sustentável. As principais estratégias de implementação incluem:
\begin{itemize}
    \item \textbf{Integração Tecnológica:} Desenvolvimento de interface para comunicação direta entre motoristas e passageiros, utilizando dados em tempo real para otimização de rotas \cite{willemann2024}.
    \item \textbf{Campanhas de Conscientização e Educação:} Promover palestras, workshops e a divulgação de materiais informativos que esclareçam as vantagens econômicas e sociais do compartilhamento de caronas, conforme enfatiza \cite{bonaldo2021}. Tais iniciativas podem ajudar a desmistificar preconceitos e incentivar a confiança no sistema.
    \item \textbf{Incentivos Institucionais:} Estabelecer parcerias com a instituição e oferecer benefícios, como descontos em serviços ou prioridade em vagas de estacionamento para os usuários frequentes do aplicativo, de forma a reconhecer e valorizar o engajamento da comunidade.
\end{itemize}

Diante dos desafios apresentados, evidencia-se a necessidade de uma solução baseada em tecnologia da informação que seja capaz de estruturar, automatizar e tornar confiável o processo de compartilhamento de caronas no contexto do IF Goiano – Câmpus Urutaí. Nesse sentido, o próximo capítulo apresenta os fundamentos tecnológicos e a arquitetura do sistema desenvolvido, bem como as decisões de projeto adotadas para atender às demandas identificadas.