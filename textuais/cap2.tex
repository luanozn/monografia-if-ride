\chapter{METODOLOGIA}

Este capítulo descreve os procedimentos técnicos e metodológicos utilizados para o desenvolvimento do \textbf{IF Ride}. O projeto é fundamentado na engenharia de software, para o desenvolvimento do aplicativo e arquitetura.

\section{Natureza e abordagem de pesquisa}

A presente pesquisa classifica-se como \textbf{aplicada}, visto que tem como objetivo auxiliar na resolução do problema do deslocamento entre as cidades vizinhas e o Câmpus Urutaí através da tecnologia.
Quanto aos objetivos, trata-se de uma pesquisa \textbf{experimental}, consistindo no desenvolvimento e teste de um protótipo de software para validação de hipóteses de facilitação de caronas.

\section{Estratégia de desenvolvimento}

Neste projeto, foi aplicada a estratégia de desenvolvimento ágil, onde foi feita a separação das responsabilidades, definição de MVPs \textit{(Minimum Viable Products)}, que são o mínimo produto viável para uma entrega, para cada uma das funcionalidades propostas, e a partir desses \textit{MVPs}, foi possível refinar as funcionalidades e trazer mais detalhes e implementações. 

\section{Arquitetura do Sistema}

O desenvolvimento do IF Ride adota o modelo de arquitetura cliente-servidor, uma estrutura distribuída que separa as tarefas entre os provedores de recursos ou serviços (servidores) e os requerentes de serviços (clientes). Esta escolha visa garantir o desacoplamento entre a interface do usuário e a lógica de negócio, facilitando a manutenção e a evolução independente de cada camada.


\section{Servidor (Back-End)}
A arquitetura do servidor foi implementada em Java utilizando o framework Spring Boot, que fornece suporte nativo para o desenvolvimento de APIs RESTful. O REST (\textit{Representational State Transfer}) é um estilo arquitetural definido por \citeonline{fielding2000} que estabelece princípios fundamentais para sistemas distribuídos.

Segundo a \citeonline{awsrest}, APIs RESTful são interfaces que permitem a troca padronizada de informações entre sistemas distribuídos, podendo ser combinadas com mecanismos de segurança, como autenticação e criptografia, para garantir a proteção dos dados transmitidos. Esta implementação destaca-se pelos seguintes pontos fortes técnicos:

\begin{enumerate}
    \item \textbf{Interface uniforme e padronização}: As operações de CRUD são mapeadas rigorosamente para os verbos HTTP (GET, POST, PUT e DELETE), utilizando JSON como formato de intercâmbio. Isso garante a interoperabilidade total com o front-end em Flutter;
    \item \textbf{Segurança e Escalabilidade (Stateless)}: O uso de tokens JWT (\textit{JSON Web Token}) em uma arquitetura \textit{stateless} elimina a necessidade de sessões no servidor, permitindo que o sistema escale horizontalmente com facilidade;
    \item \textbf{Separação de Responsabilidades (Layered System)}: A organização em camadas (\textit{Controller}, \textit{Service} e \textit{Repository}) isola a lógica de negócio. Isso permite que a regra de aprovação de motoristas seja testada e mantida de forma independente da persistência de dados;
    \item \textbf{Gestão de Estado de Segurança}: O servidor atua como o validador central do estado do motorista. Diferente de sistemas genéricos, o IF Ride bloqueia nativamente a criação de caronas por usuários cujos documentos ainda não foram validados por um administrador no módulo de gestão;
    \item \textbf{Integração de Infraestrutura em Nuvem}: O uso de serviços AWS, como o S3 para armazenamento de documentos, demonstra uma arquitetura moderna que delega a gestão de arquivos binários para serviços especializados, focando o código Java estritamente na lógica do domínio.
\end{enumerate}

\subsubsection{Arquitetura}
A arquitetura desenvolvida tem como estrutura, a AWS \textit{(Amazon Web Services)}, que segundo a própria \cite{awswhatis2025}, é uma provedora de infraestrutura e serviços de computação em nuvem

A Figura~\ref{fig:arquitetura} apresenta o diagrama da arquitetura proposta, evidenciando a interação entre os principais componentes da solução.

\begin{figure}[H]
  \centering
  \caption{Diagrama da arquitetura da aplicação em nuvem}
  \includegraphics[width=0.9\textwidth]{figuras/diagramas/arquitetura/arquitetura.png}
  \label{fig:arquitetura}
  \fonte{Elaboração própria}
\end{figure}

A arquitetura proposta tem como objetivo principal garantir a segurança, a escalabilidade e a eficiência no processamento das requisições HTTP originadas pelo front-end móvel da aplicação. Conforme ilustrado na Figura~\ref{fig:arquitetura}, a solução adota uma arquitetura baseada em serviços gerenciados em nuvem, reduzindo a complexidade operacional e os custos de infraestrutura.

O fluxo de comunicação inicia-se no cliente móvel, que realiza requisições à API por meio do Amazon API Gateway, configurado como ponto de entrada da aplicação. Antes que qualquer requisição seja encaminhada à camada de aplicação, o API Gateway realiza a validação do token JWT por meio de um autorizador nativo, assegurando que apenas requisições autenticadas sejam processadas.

A utilização de um autorizador JWT no Amazon API Gateway, mesmo com a presença de mecanismos de autenticação e autorização na camada de aplicação, fundamenta-se no princípio de defesa em profundidade. A validação antecipada do token permite que requisições inválidas ou malformadas sejam bloqueadas antes de alcançarem o AWS App Runner, evitando o provisionamento desnecessário de instâncias para processamento de chamadas não autorizadas.

Essa estratégia é particularmente relevante em cenários de alta volumetria de requisições ou potenciais ataques de negação de serviço em nível de aplicação, nos quais o escalonamento automático do App Runner poderia resultar em aumento significativo de custos e degradação do serviço. Ao concentrar a validação inicial no API Gateway, que opera de forma altamente escalável e com menor custo, a arquitetura torna-se mais eficiente, resiliente e economicamente sustentável.

Após a autenticação, as requisições válidas são direcionadas à aplicação hospedada no AWS App Runner, responsável pela execução da lógica de negócio. O uso de um serviço gerenciado e sob demanda possibilita o provisionamento automático de recursos conforme a carga de trabalho, contribuindo para a otimização de custos, uma vez que instâncias são alocadas apenas quando necessário.

A camada de persistência de dados é composta por um banco de dados relacional PostgreSQL, gerenciado pelo Amazon RDS, que oferece alta disponibilidade, backups automatizados e gerenciamento simplificado. A aplicação mantém comunicação direta com o banco de dados para armazenamento e recuperação das informações persistentes.

Todos os componentes da arquitetura encontram-se inseridos em uma Virtual Private Cloud (VPC), que atua como uma camada adicional de segurança ao restringir o acesso direto aos serviços internos. Essa configuração garante maior isolamento da infraestrutura, limitando a exposição dos recursos à internet pública e reduzindo a superfície de ataque da aplicação.

A arquitetura também prevê a separação dos ambientes de desenvolvimento e produção, permitindo a validação de novas funcionalidades em um ambiente controlado antes de sua disponibilização em produção. Essa separação contribui para a estabilidade da aplicação e reduz riscos associados a alterações no sistema.