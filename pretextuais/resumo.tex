%--------------------------------------------------------------------------
%		Não é necessário mudar qualquer coisa no preambulo deste arquivo
%		basta apenas colocar o seu texto, e as palavras-chave
%		observando as normas ABNT
%--------------------------------------------------------------------------
\chapter*{RESUMO} 
\thispagestyle{empty}
O presente Trabalho de Conclusão de Curso tem como objetivo o desenvolvimento de uma API para gerenciamento de \textit{tickets} de atendimento, oferecendo uma solução eficiente, flexível e customizável para empresas que necessitam organizar e otimizar seus fluxos de suporte ao cliente. O trabalho justifica-se pela crescente demanda por ferramentas que melhorem a comunicação entre cliente e suporte, reduzindo falhas operacionais e promovendo maior agilidade e controle nos atendimentos. A metodologia adotada para o desenvolvimento com foco em entregas funcionais, utilizando a linguagem Java com o framework Spring Boot, banco de dados MySQL, autenticação via JWT, cache e documentação da API por meio do Swagger. A API implementa funcionalidades como autenticação segura via JWT, notificações automáticas, fluxos inteligentes de triagem e acompanhamento de chamados, além de ser construída com uma arquitetura modular e escalável, favorecendo a manutenção e a integração com outros sistemas corporativos. 
 A solução proposta visa contribuir diretamente para a satisfação do cliente e para a eficiência dos processos internos das empresas.



 
 
%--------------------------------------------------------------------------
\vspace{2cm}
{\Large \textbf{Palavras-chave:}}
\vspace{0.4cm}
%--------------------------------------------------------------------------

Spring Boot, API REST, Gerenciamento de \textit{Tickets}, JSON Web Token (JWT)
%MAIS DICAS
%
%Passo 1: O que é (contexto)
%
%Para iniciar o resumo e abstract, apresente ao leitor do que se trata seu trabalho, ou seja, qual problema você está investigando e a sua relevância.
%
%Logo aqui, nas primeiras linhas, você precisa criar um contexto que desperte a curiosidade e o interesse para a leitura.
%
%Passo 2: Por quê? (Descreva o objetivo do trabalho)
%
%Com base no problema ou situação investigada, seu trabalho já precisa dizer a que veio. A forma de descrever um objetivo varia de acordo com o estudo realizado.
%
%Seu trabalho pode se propor a analisar diferentes aspectos de um mesmo problema, a compreender um fenômeno, ou então, a “apresentar uma solução para reduzir a produção de energia não renovável”, por exemplo. Em resumo, você conta o porquê do seu TCC existir.
%
%Passo 3: Como? (Método utilizado)
%
%Você deve explicar rapidamente como seu trabalho foi feito, qual metodologia utilizou. Conte ao seu leitor se você realizou um estudo de caso, um experimento, uma pesquisa quantitativa e/ou pesquisa qualitativa etc.
%
%Passo 4: Resultados
%
%Apresente os resultados mais relevantes do seu estudo, sejam eles positivos ou negativos. Descreva-os de forma sucinta, e que despertem curiosidade e interesse em quem lê.
%
%Passo 5: Conclui-se que…
%
%Por fim, seu resumo deve trazer as principais conclusões sobre tudo o que você leu, pesquisou e desenvolveu.
%
%Aqui, também é possível mencionar como o seu trabalho poderá contribuir para o entendimento do objeto ou fenômeno estudado, ou como servirá de base para pesquisas futuras.
%\vspace{-1cm}


