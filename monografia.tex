%-------------------------------------------------------------------------
% Modelo de monografia
% confeccionado em março de 2019, 
% by Cristiane de Fátima dos Santos Cardoso
% plataforma: windows 7, miktex 2.9.48, TexMaker 4.1
% versao 2.0
%------------------------------------------------------------------------
%\documentclass[12pt,a4paper, chapter=TITLE,section=TITLE,alf,oneside]{abntex2} %estilo abnt um lado
%------------------------------------------------------------------------
%se desejar a impressão frente e verso retire o simbolo de porccentagem da linha abaixo e coloque na linha de cima. Observe que a impressão frente e verso gera diversas folhas em branco, isso não é uma falha!!!! Capítulos sempre devem ser iniciados na próxima folha a direita!
%------------------------------------------------------------------------
%\documentclass[12pt,a4paper,alf]{abntex2} % estilo abnt frente e verso
\documentclass[12pt, a4paper]{report}
%------------------------------------------------------------------------

\usepackage{formatacao}      
\usepackage{tabularx}
\usepackage{graphicx}     
\usepackage{pdfpages}
% estilo proprio
\makenomenclature                          % construir lista de acronimos
%
%--------------------------------------------------------------------
% DEFINIÇÕES
%           Fornecendo as informações abaixo, automaticamente serão 
%			geradas: capa e a seção de identificação
%---------------------------------------------------------------------
%IMPORTANTE: caso haja dois autores faça como no exemplo
%	\def\autor{João da Silva e Maria da Silva}						
%---------------------------------------------------------------------

\def\autor{Bruno César Vieira Fernandes}                		 %Autor
\def\autordois{Luan de Castro Ribeiro}
\def\titulo{DESENVOLVIMENTO DO APLICATIVO IF RIDE PARA FACILITAÇÃO DE CARONAS NO INSTITUTO FEDERAL GOIANO - CÂMPUS URUTAÍ} 				 %Titulo em letra maiúscula
\def\instituicao{Instituto Federal Goiano}
\def\campus{Campus Urutaí} 
\def\unidade{Núcleo de Informática}   			 %Unidade 
\def\curso{Sistemas de Informação}


\def\orientador{Junio César Vieira de Lima}            

\def\titulacaoorientador{Dr.} 

\def\avaliadorum{Professor 1}
\def\titulacaoavaliadorum{PhD.}
\def\avaliadordois{Professor 2}
\def\titulacaoavaliadordois{Dr.}

\def\local{Urutaí}
\def\datadefesa{23 de agosto de 2024} 
\def\tipo{Monografia}%Dissertação
\def\qualificacao{Bacharel}%Licenciatura
\def\area{\curso}
\def\tema{confecção de Trabalho de Curso}
\def\data{\today} 
\makeglossary
%------------------------------------------------------------ 	
\begin{document}
%------------------------------------------------------------
% ELEMENTOS PRE-TEXTUAIS
%		não é necessário editar capa, folha de rosto e folha 
%		de aprovação! para editar o resumo entre na pasta 
%		pretextuais e localize o arquivo resumo.tex, editando-o 
%		conforme as dicas, faça o mesmo com fichacatalografica.tex,
%		dedicatoria.tex, agradecimentos.tex, epigrafe.tex. 
%		Acrescente um % na frente de elementos opcionais que 
%		não desejar na sua monografia
%--------------------------------------------------------------
\thispagestyle{empty}
\begin{center}

%------------PRIMEIRA FIGURA------------------
\vspace{-0.9cm}
\includegraphics[scale=0.45]{figuras/logo_oficial.png} 
\bigskip
%---------------------------------------------
\hfill

%-------------SEGUNDA FIGURA(TEXTO)-----------
\begin{center}
\begin{large}
\textbf{\textsc{\instituicao}}\\
\textbf{\textsc\campus}\\
\small
\textsc{\unidade}\\
\textsc{Curso de \curso}\\
\medskip
\end{large}
\end{center}
%--------------------------------------------
\hfil

\end{center}

\vspace{1 cm}
\begin{Large}
\begin{center}
\textsc\autor\\
\textsc\autordois\
\vspace{3cm}

\LARGE\textbf{\titulo}
\end{center}
\end{Large}

\vspace{3cm}
\vfill

\begin{center}
\local, \data
\end{center}



                   %inserindo a capa
%---------------------------------------------------------------------
\clearpage{\pagestyle{empty}\cleardoublepage %contracapa
\setcounter{page}{1}                         %começa a contar daqui
%\renewcommand{\thepage}{\roman{page}}        %o tipo de numeração    
\pagenumbering{arabic}
\pagestyle{empty}

\setcounter{page}{1}   

\includepdf[pages=-]{pretextuais/ficha-catalografica.pdf}

\begin{center}
    \includegraphics[width=\textwidth,height=0.9\textheight]{pretextuais/TCAE_-_Termo_de_autorizacao_1.jpg}
\end{center}

\begin{center}
    \includegraphics[width=\textwidth,height=0.9\textheight]{pretextuais/TCAE_-_Termo_de_autorizacao_2.jpg}
\end{center}


\includepdf[pages=-]{pretextuais/ata_defesa.pdf}


\includepdf[pages=-]{pretextuais/folhaaprovacao_diegofelipe.pdf}

%\pagestyle{empty}
\begin{agradecimentos}

Agradeço primeiramente a Deus, pela força, saúde e sabedoria concedidas durante toda minha caminhada acadêmica. Aos meus pais e familiares, pelo amor, apoio incondicional e incentivo contínuo em todas as etapas da minha vida.Aos colegas de turma e amigos que estiveram presentes durante essa jornada, pelas trocas de experiências, pelas conversas e pelas colaborações que tornaram essa caminhada mais leve e significativa.
E, por fim, agradeço à instituição pela estrutura e oportunidade de aprendizado que me proporcionou ao longo da graduação.


\end{agradecimentos}
       %agradecimentos-opcional
%%As chaves abaixo não devem ser retiradas
\epigrafe
{É uma citação relacionada ao assunto do texto}
{Autor da citação}
{Obra de onde foi retirada a citação}
             %epigrafe-opcional
%--------------------------------------------------------------------------
%		Não é necessário mudar qualquer coisa no preambulo deste arquivo
%		basta apenas colocar o seu texto, e as palavras-chave
%		observando as normas ABNT
%--------------------------------------------------------------------------
\chapter*{RESUMO} 
\thispagestyle{empty}
O presente Trabalho de Conclusão de Curso tem como objetivo o desenvolvimento de uma API para gerenciamento de \textit{tickets} de atendimento, oferecendo uma solução eficiente, flexível e customizável para empresas que necessitam organizar e otimizar seus fluxos de suporte ao cliente. O trabalho justifica-se pela crescente demanda por ferramentas que melhorem a comunicação entre cliente e suporte, reduzindo falhas operacionais e promovendo maior agilidade e controle nos atendimentos. A metodologia adotada para o desenvolvimento com foco em entregas funcionais, utilizando a linguagem Java com o framework Spring Boot, banco de dados MySQL, autenticação via JWT, cache e documentação da API por meio do Swagger. A API implementa funcionalidades como autenticação segura via JWT, notificações automáticas, fluxos inteligentes de triagem e acompanhamento de chamados, além de ser construída com uma arquitetura modular e escalável, favorecendo a manutenção e a integração com outros sistemas corporativos. 
 A solução proposta visa contribuir diretamente para a satisfação do cliente e para a eficiência dos processos internos das empresas.



 
 
%--------------------------------------------------------------------------
\vspace{2cm}
{\Large \textbf{Palavras-chave:}}
\vspace{0.4cm}
%--------------------------------------------------------------------------

Spring Boot, API REST, Gerenciamento de \textit{Tickets}, JSON Web Token (JWT)
%MAIS DICAS
%
%Passo 1: O que é (contexto)
%
%Para iniciar o resumo e abstract, apresente ao leitor do que se trata seu trabalho, ou seja, qual problema você está investigando e a sua relevância.
%
%Logo aqui, nas primeiras linhas, você precisa criar um contexto que desperte a curiosidade e o interesse para a leitura.
%
%Passo 2: Por quê? (Descreva o objetivo do trabalho)
%
%Com base no problema ou situação investigada, seu trabalho já precisa dizer a que veio. A forma de descrever um objetivo varia de acordo com o estudo realizado.
%
%Seu trabalho pode se propor a analisar diferentes aspectos de um mesmo problema, a compreender um fenômeno, ou então, a “apresentar uma solução para reduzir a produção de energia não renovável”, por exemplo. Em resumo, você conta o porquê do seu TCC existir.
%
%Passo 3: Como? (Método utilizado)
%
%Você deve explicar rapidamente como seu trabalho foi feito, qual metodologia utilizou. Conte ao seu leitor se você realizou um estudo de caso, um experimento, uma pesquisa quantitativa e/ou pesquisa qualitativa etc.
%
%Passo 4: Resultados
%
%Apresente os resultados mais relevantes do seu estudo, sejam eles positivos ou negativos. Descreva-os de forma sucinta, e que despertem curiosidade e interesse em quem lê.
%
%Passo 5: Conclui-se que…
%
%Por fim, seu resumo deve trazer as principais conclusões sobre tudo o que você leu, pesquisou e desenvolveu.
%
%Aqui, também é possível mencionar como o seu trabalho poderá contribuir para o entendimento do objeto ou fenômeno estudado, ou como servirá de base para pesquisas futuras.
%\vspace{-1cm}


               %resumo
%\include{pretextuais/abstract}             %resumo

\clearpage{\pagestyle{empty}\cleardoublepage %verso do resumo em branco



\pagestyle{plain}

\renewcommand{\thepage}{\roman{page}}
\clearpage\listoffigures                        %lista de figuras - opcional
\newpage
\listoftables                        		    %lista de tabelas - opcional
\newpage
%\listofalgorithms                               %lista de algoritmos - opciona
\renewcommand{\nomname}{Lista de abreviaturas e siglas}     %renomeando lista de acronimos
\renewcommand{\pagedeclaration}{Pag.}            %renomeando lista de acronimos
\printnomenclature	\label{listaacronimos}						    %gerando a lista de acronimos
\newpage
\tableofcontents   							    %inserindo o sumário
\clearpage
%-----------------------------------------------------------------
% ELEMENTOS TEXTUAIS
%		localize cada um dos arquivos na pasta textuais
%       e edite da forma que desejar, caso necessite de 
% 		mais capitulos  é só criar os arquivos e incluir
%-----------------------------------------------------------------
\setlength{\baselineskip}{1.5\baselineskip}  %espaçamento entrelinhas 1.5
\pagestyle{fancy}							 %tipo de estilo da página - cabeçalho e rodapé
\renewcommand{\thepage}{\arabic{page}}
\citeoption{abnt-etal-cite=2}           % a partir de 2 autores, usa o et al.

\chapter{Fundamentos básicos} \label{fundamentos_basicos}

\section{Câmpus}

O \textit{IF Goiano - Câmpus Urutaí} é uma instituição de ensino localizada na zona rural, instalada em uma área de 512 hectares na região da Estrada de Ferro, no sudeste do Estado de Goiás \cite{ifgoiano2024}. Fundado a partir de uma rica trajetória histórica que remonta à criação da Fazenda Modelo de Criação em 1918 – um marco que possibilitou a formação de profissionais para o setor agropecuário –, o campus evoluiu de Escola Agrícola para Escola Agrotécnica e, posteriormente, para a sua atual configuração como Instituto Federal Goiano \cite{turn0search1}. Ao longo de sua história, o campus tem desempenhado um papel fundamental na democratização do acesso à educação técnica, profissional e superior, contribuindo para o desenvolvimento regional e para a formação de profissionais qualificados que atendem às demandas do mercado local \cite{turn0search2}. Além disso, o IF Goiano - Campus Urutaí destaca-se pela integração entre ensino, pesquisa e extensão, fortalecendo seu compromisso com a inovação e a transformação social, aspectos que se refletem na oferta de uma ampla gama de cursos – desde o ensino médio integrado até programas de pós-graduação \cite{turn0search6}. Essa diversidade de modalidades de ensino ressalta a importância do campus como polo de desenvolvimento educacional e social na região.


\section{Principais desafios}

Uma das principais dificuldades enfrentadas pelos estudantes e servidores do Instituto Federal Goiano, especialmente no Câmpus Urutaí, refere-se à sua localização geográfica em área rural, caracterizada pelo acesso limitado a opções de transporte público e pela distância significativa dos centros urbanos. A ausência de transporte público nessa região, aliada ao fato de que
as cidades que compõem o entorno do câmpus também são bastante pequenas, resulta na falta de
uma alternativa eficiente ao transporte privado. Infelizmente, o transporte privado geralmente
é excessivamente caro, propenso a atrasos e não confiável, o que acaba criando um cenário
de monopólio para os usuários e limitando suas opções de escolha. \cite{propostaTCC2025}.

A situação do transporte no câmpus apresenta-se como um desafio significativo para a comunidade acadêmica. Conforme mencionado anteriormente, os custos associados ao transporte são elevados, o que impacta diretamente estudantes que enfrentam condições financeiras adversas. Muitos desses alunos, diante da dificuldade de arcar com o deslocamento diário, optam por residir nas dependências da instituição para viabilizar a continuidade dos estudos.
Embora algumas prefeituras ofereçam auxílios municipais destinados ao transporte até o Instituto, e existam bolsas de assistência estudantil para aqueles que comprovam baixa renda per capita, tais medidas ainda se mostram insuficientes. O auxílio municipal, por si só, não é capaz de reduzir significativamente os custos do transporte privado, mantendo-o inacessível para uma parcela considerável dos alunos. Além disso, as bolsas disponibilizadas não contemplam todos os estudantes em situação de vulnerabilidade, sendo limitadas tanto em número quanto em valor.
Diante desse cenário, é evidente que a dificuldade de acesso afeta não apenas os estudantes das cidades que recebem algum tipo de auxílio, mas agrava-se ainda mais para aqueles provenientes de municípios não contemplados. Esse contexto contribui para o aumento da evasão escolar e dificulta o acesso ao ensino para alunos de baixa renda, comprometendo a inclusão e a permanência no Instituto.


\section{Mobilidade Compartilhada: Conceito e Relevância}
De acordo com \cite{bonaldo2021}, a mobilidade compartilhada refere-se à utilização coletiva e otimizada de meios de transporte, apoiada em tecnologias de informação que possibilitam a integração de diferentes modais. Ele propõe que a mudança do paradigma da propriedade individual para o compartilhamento de veículos pode revolucionar o planejamento urbano, reduzindo custos e otimizando a infraestrutura das cidades. Essa perspectiva é reforçada por \cite{willemann2024}, que enfatiza as inovações tecnológicas como fator determinante para a viabilização de sistemas colaborativos de transporte. A autora também argumenta que a convergência entre tecnologia e mobilidade cria oportunidades para cidades inteligentes, onde os dados em tempo real permitem a melhor distribuição dos recursos disponíveis. Em suma, ambos os autores apontam para a necessidade de repensar o transporte urbano de forma integrada e sustentável, destacando que a mobilidade compartilhada não se trata apenas de uma alternativa econômica, mas de uma estratégia de transformação social.

\subsection{Impactos Ambientais da Mobilidade Compartilhada}
O benefício ambiental da mobilidade compartilhada reside na otimização da taxa de ocupação dos veículos. Conforme \citeonline{moro2022}, a redução do volume de veículos particulares em circulação correlaciona-se diretamente com a diminuição das emissões de gases poluentes. Do ponto de vista técnico, sistemas de caronas permitem que a mesma demanda de deslocamento seja suprida com menor consumo energético por passageiro. Além da redução de emissões, \citeonline{willemann2024} aponta que o fluxo de tráfego mais eficiente mitiga o tempo de ociosidade de motores em congestionamentos, potencializando a economia de combustível e a redução de custos operacionais para os usuários.

\subsection{Incentivos à Adoção da Mobilidade Compartilhada}
Para que a mobilidade compartilhada se consolide, é fundamental o desenvolvimento de políticas públicas e estratégias de incentivo. \cite{bonaldo2021} destaca que medidas como subsídios, investimentos em infraestrutura (por exemplo, ciclovias e pontos de compartilhamento) e campanhas de conscientização são essenciais para superar barreiras culturais e econômicas. \cite{bonaldo2021} enfatiza que a integração entre os setores público e privado pode facilitar a implementação de sistemas interconectados, aumentando a confiabilidade e a adesão dos usuários. Essa integração, segundo o autor, não apenas reduz os custos individuais com transporte, mas também promove uma maior inclusão social, ao oferecer alternativas acessíveis a diferentes faixas da população.

\subsection{Tecnologia, Segurança e Outras Considerações Importantes}
Os desafios da mobilidade compartilhada não se restringem apenas à sua implementação operacional, mas também envolvem aspectos tecnológicos e de segurança. \cite{moro2022} discute a necessidade de investimentos contínuos em sistemas de monitoramento, interfaces intuitivas e protocolos de segurança robustos, os quais são cruciais para garantir a confiança dos usuários. Os autores argumentam que a interoperabilidade entre plataformas e a regulamentação adequada são determinantes para a expansão sustentável desse modelo. Outro ponto importante refere-se à adaptação da infraestrutura urbana, que deve acompanhar as novas demandas geradas pelos serviços de mobilidade compartilhada, proporcionando espaços seguros e bem planejados para veículos e pedestres.


\subsection{A Mobilidade Compartilhada no IF Goiano - Campus Urutaí}

No contexto do Câmpus Urutaí, a mobilidade compartilhada atua como uma solução para a carência de transporte público em áreas rurais. A utilização de caronas entre a comunidade acadêmica permite otimizar os deslocamentos, reduzindo o custo logístico para estudantes e servidores. Atualmente, o compartilhamento ocorre de forma fragmentada, muitas vezes dependendo de redes sociais externas ou acordos informais.

A implementação de uma plataforma dedicada visa formalizar e automatizar esse processo. Segundo as definições de \citeonline{bonaldo2021} e \citeonline{willemann2024}, o uso de tecnologias de informação é o que permite a integração eficiente desses modais. Para o câmpus, o desenvolvimento de um aplicativo de caronas resolve o problema do alto valor das tarifas privadas, permitindo a divisão de custos de forma auditável e sustentável. As principais estratégias de implementação incluem:
\begin{itemize}
    \item \textbf{Integração Tecnológica:} Desenvolvimento de interface para comunicação direta entre motoristas e passageiros, utilizando dados em tempo real para otimização de rotas \cite{willemann2024}.
    \item \textbf{Campanhas de Conscientização e Educação:} Promover palestras, workshops e a divulgação de materiais informativos que esclareçam as vantagens econômicas e sociais do compartilhamento de caronas, conforme enfatiza \cite{bonaldo2021}. Tais iniciativas podem ajudar a desmistificar preconceitos e incentivar a confiança no sistema.
    \item \textbf{Incentivos Institucionais:} Estabelecer parcerias com a instituição e oferecer benefícios, como descontos em serviços ou prioridade em vagas de estacionamento para os usuários frequentes do aplicativo, de forma a reconhecer e valorizar o engajamento da comunidade.
\end{itemize}

Diante dos desafios apresentados, evidencia-se a necessidade de uma solução baseada em tecnologia da informação que seja capaz de estruturar, automatizar e tornar confiável o processo de compartilhamento de caronas no contexto do IF Goiano – Câmpus Urutaí. Nesse sentido, o próximo capítulo apresenta os fundamentos tecnológicos e a arquitetura do sistema desenvolvido, bem como as decisões de projeto adotadas para atender às demandas identificadas.                 
\chapter{METODOLOGIA}

Este capítulo descreve os procedimentos técnicos e metodológicos utilizados para o desenvolvimento do \textbf{IF Ride}. O projeto é fundamentado na engenharia de software, para o desenvolvimento do aplicativo e arquitetura.

\section{Natureza e abordagem de pesquisa}

A presente pesquisa classifica-se como \textbf{aplicada}, visto que tem como objetivo auxiliar na resolução do problema do deslocamento entre as cidades vizinhas e o Câmpus Urutaí através da tecnologia.
Quanto aos objetivos, trata-se de uma pesquisa \textbf{experimental}, consistindo no desenvolvimento e teste de um protótipo de software para validação de hipóteses de facilitação de caronas.

\section{Estratégia de desenvolvimento}

Neste projeto, foi aplicada a estratégia de desenvolvimento ágil, onde foi feita a separação das responsabilidades, definição de MVPs \textit{(Minimum Viable Products)}, que são o mínimo produto viável para uma entrega, para cada uma das funcionalidades propostas, e a partir desses \textit{MVPs}, foi possível refinar as funcionalidades e trazer mais detalhes e implementações. 

\section{Arquitetura do Sistema}

O desenvolvimento do IF Ride adota o modelo de arquitetura cliente-servidor, uma estrutura distribuída que separa as tarefas entre os provedores de recursos ou serviços (servidores) e os requerentes de serviços (clientes). Esta escolha visa garantir o desacoplamento entre a interface do usuário e a lógica de negócio, facilitando a manutenção e a evolução independente de cada camada.


\section{Servidor (Back-End)}
A arquitetura do servidor foi implementada em Java utilizando o framework Spring Boot, que fornece suporte nativo para o desenvolvimento de APIs RESTful. O REST (\textit{Representational State Transfer}) é um estilo arquitetural definido por \citeonline{fielding2000} que estabelece princípios fundamentais para sistemas distribuídos.

Segundo a \citeonline{awsrest}, APIs RESTful são interfaces que permitem a troca padronizada de informações entre sistemas distribuídos, podendo ser combinadas com mecanismos de segurança, como autenticação e criptografia, para garantir a proteção dos dados transmitidos. Esta implementação destaca-se pelos seguintes pontos fortes técnicos:

\begin{enumerate}
    \item \textbf{Interface uniforme e padronização}: As operações de CRUD são mapeadas rigorosamente para os verbos HTTP (GET, POST, PUT e DELETE), utilizando JSON como formato de intercâmbio. Isso garante a interoperabilidade total com o front-end em Flutter;
    \item \textbf{Segurança e Escalabilidade (Stateless)}: O uso de tokens JWT (\textit{JSON Web Token}) em uma arquitetura \textit{stateless} elimina a necessidade de sessões no servidor, permitindo que o sistema escale horizontalmente com facilidade;
    \item \textbf{Separação de Responsabilidades (Layered System)}: A organização em camadas (\textit{Controller}, \textit{Service} e \textit{Repository}) isola a lógica de negócio. Isso permite que a regra de aprovação de motoristas seja testada e mantida de forma independente da persistência de dados;
    \item \textbf{Gestão de Estado de Segurança}: O servidor atua como o validador central do estado do motorista. Diferente de sistemas genéricos, o IF Ride bloqueia nativamente a criação de caronas por usuários cujos documentos ainda não foram validados por um administrador no módulo de gestão;
    \item \textbf{Integração de Infraestrutura em Nuvem}: O uso de serviços AWS, como o S3 para armazenamento de documentos, demonstra uma arquitetura moderna que delega a gestão de arquivos binários para serviços especializados, focando o código Java estritamente na lógica do domínio.
\end{enumerate}

\subsubsection{Arquitetura}
A arquitetura desenvolvida tem como estrutura, a AWS \textit{(Amazon Web Services)}, que segundo a própria \cite{awswhatis2025}, é uma provedora de infraestrutura e serviços de computação em nuvem

A Figura~\ref{fig:arquitetura} apresenta o diagrama da arquitetura proposta, evidenciando a interação entre os principais componentes da solução.

\begin{figure}[H]
  \centering
  \caption{Diagrama da arquitetura da aplicação em nuvem}
  \includegraphics[width=0.9\textwidth]{figuras/diagramas/arquitetura/arquitetura.png}
  \label{fig:arquitetura}
  \fonte{Elaboração própria}
\end{figure}

A arquitetura proposta tem como objetivo principal garantir a segurança, a escalabilidade e a eficiência no processamento das requisições HTTP originadas pelo front-end móvel da aplicação. Conforme ilustrado na Figura~\ref{fig:arquitetura}, a solução adota uma arquitetura baseada em serviços gerenciados em nuvem, reduzindo a complexidade operacional e os custos de infraestrutura.

O fluxo de comunicação inicia-se no cliente móvel, que realiza requisições à API por meio do Amazon API Gateway, configurado como ponto de entrada da aplicação. Antes que qualquer requisição seja encaminhada à camada de aplicação, o API Gateway realiza a validação do token JWT por meio de um autorizador nativo, assegurando que apenas requisições autenticadas sejam processadas.

A utilização de um autorizador JWT no Amazon API Gateway, mesmo com a presença de mecanismos de autenticação e autorização na camada de aplicação, fundamenta-se no princípio de defesa em profundidade. A validação antecipada do token permite que requisições inválidas ou malformadas sejam bloqueadas antes de alcançarem o AWS App Runner, evitando o provisionamento desnecessário de instâncias para processamento de chamadas não autorizadas.

Essa estratégia é particularmente relevante em cenários de alta volumetria de requisições ou potenciais ataques de negação de serviço em nível de aplicação, nos quais o escalonamento automático do App Runner poderia resultar em aumento significativo de custos e degradação do serviço. Ao concentrar a validação inicial no API Gateway, que opera de forma altamente escalável e com menor custo, a arquitetura torna-se mais eficiente, resiliente e economicamente sustentável.

Após a autenticação, as requisições válidas são direcionadas à aplicação hospedada no AWS App Runner, responsável pela execução da lógica de negócio. O uso de um serviço gerenciado e sob demanda possibilita o provisionamento automático de recursos conforme a carga de trabalho, contribuindo para a otimização de custos, uma vez que instâncias são alocadas apenas quando necessário.

A camada de persistência de dados é composta por um banco de dados relacional PostgreSQL, gerenciado pelo Amazon RDS, que oferece alta disponibilidade, backups automatizados e gerenciamento simplificado. A aplicação mantém comunicação direta com o banco de dados para armazenamento e recuperação das informações persistentes.

Todos os componentes da arquitetura encontram-se inseridos em uma Virtual Private Cloud (VPC), que atua como uma camada adicional de segurança ao restringir o acesso direto aos serviços internos. Essa configuração garante maior isolamento da infraestrutura, limitando a exposição dos recursos à internet pública e reduzindo a superfície de ataque da aplicação.

A arquitetura também prevê a separação dos ambientes de desenvolvimento e produção, permitindo a validação de novas funcionalidades em um ambiente controlado antes de sua disponibilização em produção. Essa separação contribui para a estabilidade da aplicação e reduz riscos associados a alterações no sistema.
\chapter{Levantamento e Análise de requisitos}

Nesta seção, são detalhadas as etapas iniciais para dar início ao desenvolvimento do sistema. O processo foi estruturado em fases que incluem o levantamento de requisitos, análise de requisitos, implementação, testes, correções e entrega. O fluxo de desenvolvimento seguiu um passo a passo lógico, garantindo uma abordagem organizada e eficiente.

\section{Levantamento e Análise de Requisitos}
O levantamento de requisitos foi realizado com base nas necessidades operacionais de um sistema de gerenciamento de \textit{tickets} para otimizar processos de atendimento e suporte. O sistema visa substituir métodos manuais ou pouco integrados, garantindo rastreabilidade, segurança e eficiência na gestão de demandas.

Durante essa análise, foram identificadas diversas funcionalidades essenciais, incluindo o controle eficiente do ciclo de vida dos \textit{tickets}, a definição de prioridades para atendimento e a necessidade de registrar todo o histórico de modificações. Além disso, destacou-se a importância da autenticação segura dos usuários e da aplicação de regras de autorização para garantir que cada perfil tenha acesso apenas às informações pertinentes.

Também foi levantada a necessidade de notificações automáticas via e-mail para manter os usuários informados sobre o andamento dos \textit{tickets}, garantindo um fluxo de comunicação eficiente.

\subsection{Níveis de acesso do Usuário}
Os usuários do sistema são classificados em três categorias, cada uma com permissões específicas:

\begin{itemize}
    \item[$\bullet$] \textbf{Cliente:} Pode criar \textit{tickets}, visualizar seus próprios chamados, adicionar comentários e anexar arquivos.
    \item[$\bullet$] \textbf{Funcionário:} Pode visualizar e atender os \textit{tickets} atribuídos a ele, adicionar comentários e registrar o histórico de ações.
    \item[$\bullet$] \textbf{Gerente:} Possui controle total sobre os \textit{tickets} do seu departamento, podendo criar e gerenciar categorias, definir regras de prioridade e gerar relatórios.
\end{itemize}

\subsection{Requisitos Funcionais}

Os principais requisitos funcionais do sistema foram definidos conforme a tabela abaixo:


\begin{longtable}{|m{3.1cm}|m{2.6cm}|m{3.5cm}|m{5cm}|}
    \hline
    \textbf{Identificação} & \textbf{Classificação} & \textbf{Ator} & \textbf{Objetivo} \\
    \hline
    Realizar Cadastro & Essencial & Cliente, Funcionário & Permitir que o cliente ou funcionário realize o cadastro no sistema. \\
    \hline
    Gerenciar Departamento & Essencial & Gerente & Criar, editar e excluir departamentos que atenderão os \textit{tickets}. \\
    \hline
    Gerenciar Categoria & Essencial & Gerente & Criar, editar ou excluir categorias de \textit{tickets}. \\
    \hline
    Gerenciar Regra de Prioridade & Essencial & Gerente & Definir critérios para classificação e priorização dos \textit{tickets}. \\
    \hline
    Gerenciar \textit{Ticket} & Essencial & Cliente, Funcionário, Gerente & Criar, visualizar e modificar \textit{tickets} conforme permissões do perfil. \\
    \hline
    Registrar Histórico & Essencial & Sistema & Armazenar um registro detalhado das ações realizadas em cada \textit{ticket}. \\
    \hline
    Realizar Atendimento & Essencial & Funcionário & Atender \textit{tickets} atribuídos e executar as ações necessárias. \\
    \hline
    Fazer Comentário & Essencial & Cliente, Funcionário & Inserir comentários nos \textit{tickets} para troca de informações. \\
    \hline
    Anexar Arquivo & Importante & Cliente, Funcionário & Permitir anexação de arquivos em \textit{tickets}. \\
    \hline
    Gerar Relatório & Importante & Funcionário, Gerente & Criar relatórios sobre os \textit{tickets} e atendimentos realizados. \\
    \hline
    \caption{Tabela de requisitos funcionais}
    \label{tab:requisitosfuncionais}
\end{longtable}


\subsection{Regras de Negócio}
O sistema de gerenciamento de \textit{tickets} foi desenvolvido para oferecer controle eficiente sobre os atendimentos realizados pelas empresas. As regras de negócio estabelecidas garantem o funcionamento adequado das funcionalidades, respeitando os níveis de acesso dos usuários e a lógica dos fluxos de atendimento.

\subsubsection{1. Autenticação e Autorização}
\begin{itemize}
    \item O sistema utiliza autenticação via JWT, garantindo segurança na troca de informações.
    \item O token contém dados do usuário, incluindo seu ID e perfil de acesso (\texttt{ROLE}), sendo validado a cada requisição.
\end{itemize}

\subsubsection{2. Regras de Acesso aos \textit{Tickets}}
\begin{itemize}
    \item \textbf{Clientes:} Apenas podem acessar os \textit{tickets} que abriram.
    \item \textbf{Funcionários:} Somente podem visualizar e alterar os \textit{tickets} atribuídos a eles.
    \item \textbf{Gerentes:} Têm acesso a todos os \textit{tickets} do seu departamento.
\end{itemize}

\subsubsection{3. Fluxo de Trabalho dos \textit{Tickets}}
\begin{itemize}
    \item \textbf{Criação:}
        \begin{itemize}
            \item No momento da criação, o sistema classifica a prioridade do \textit{ticket}.
            \item É enviada uma notificação automática ao cliente e ao responsável pelo atendimento.
            \item O tempo máximo de resolução é calculado com base na prioridade e na categoria do chamado.
        \end{itemize}
\newpage
    \item \textbf{Atualização:}
        \begin{itemize}
            \item Todas as mudanças nos \textit{tickets} são registradas no histórico.
            \item O sistema verifica e registra alterações feitas nos campos do \textit{ticket}.
            \item Caso o status de um \textit{ticket} seja alterado para "EM ANDAMENTO", o cliente recebe uma notificação automática.
        \end{itemize}
    \item \textbf{Comentários:}
        \begin{itemize}
            \item Apenas usuários autorizados podem adicionar comentários.
            \item Clientes só podem comentar nos seus próprios \textit{tickets}.
            \item Funcionários só podem comentar nos \textit{tickets} pelos quais são responsáveis.
            \item Gerentes só podem comentar nos \textit{tickets} do seu departamento.
        \end{itemize}
    \item \textbf{Criação de Usuário:}
        \begin{itemize}
            \item O sistema valida os dados informados no cadastro, incluindo CPF e e-mail.
            \item Não é permitido o cadastro de usuários duplicados.
            \item A senha do usuário é encriptada antes de ser armazenada.
            \item Após o cadastro, um e-mail de confirmação é enviado ao usuário.
        \end{itemize}
\end{itemize}

\subsection{Envio de E-mails}
O sistema possui um serviço de envio de e-mails automáticos para manter os usuários informados sobre eventos relevantes. As principais notificações incluem:

\begin{itemize}
    \item \textbf{Cadastro de usuário:} Confirmação da criação da conta e instruções para ativação.
    \item \textbf{Criação de \textit{ticket}:} Notificação ao cliente e ao responsável pelo atendimento.
    \item \textbf{Atualização de status:} Aviso ao cliente sempre que o \textit{ticket} sofrer alterações.
\end{itemize}

\newpage

\subsection{Diagrama de Casos de Uso}
A Figura \ref{fig:diagramacasosdeuso} ilustra as interações entre os usuários e o sistema, evidenciando as principais funcionalidades acessíveis a cada perfil. Esse diagrama fornece uma visão geral do comportamento do sistema a partir da perspectiva dos usuários, facilitando a compreensão dos requisitos e do escopo das operações disponíveis.

\begin{figure}[h]
    \centering
    \includegraphics[scale=0.5]{figuras/diagramas/casos de uso/Diagrama_de_caso_de_uso.png}
    \caption{Diagrama de casos de uso}
    \label{fig:diagramacasosdeuso}
\end{figure} 

\newpage

\subsubsection{UC01 - Fazer Cadastro}

Antes de apresentar os casos de uso do sistema, é importante contextualizar que eles representam as interações entre os usuários (atores) e o sistema, descrevendo o comportamento esperado em diferentes situações. Cada caso de uso é descrito de forma estruturada, utilizando um modelo tabular que contempla os principais elementos, como atores envolvidos, pré e pós-condições, ações e restrições.

A Tabela~\ref{tab:uc01_fazer_cadastro} apresenta o caso de uso UC01, que descreve o processo de cadastro de um novo usuário no sistema:

\begin{table}[h]
\centering
\caption{Descrição do Caso de Uso UC01 - Fazer Cadastro}
\label{tab:uc01_fazer_cadastro}
\resizebox{\textwidth}{!}{%
\begin{tabular}{cllll}
\hline
\rowcolor[HTML]{CCCCCC} 
Nome do caso de uso                                                   & \multicolumn{4}{c}{\cellcolor[HTML]{CCCCCC}UC01 - Fazer Cadastro}               \\ \hline
\cellcolor[HTML]{CCCCCC}Ator principal                                & \multicolumn{4}{l}{{\color[HTML]{333333} Usuário}}                             \\
\cellcolor[HTML]{CCCCCC}Atores secundários                            & \multicolumn{4}{l}{-}                                                           \\
\cellcolor[HTML]{CCCCCC}Resumo                                        & \multicolumn{4}{l}{Permite que um paciente faça o cadastro no sistema.}        \\ \hline
\cellcolor[HTML]{CCCCCC}Pré-condições                                 & \multicolumn{4}{l}{-}                                                           \\
\cellcolor[HTML]{CCCCCC}Pós-condições                                 & \multicolumn{4}{l}{Usuário cadastrado recebe e-mail de confirmação.}           \\
\rowcolor[HTML]{CCCCCC} 
\multicolumn{5}{c}{\cellcolor[HTML]{CCCCCC}Cenário Principal}                  \\
\rowcolor[HTML]{CCCCCC} 
Ações do ator                                                         & \multicolumn{4}{c}{\cellcolor[HTML]{CCCCCC}Ações do sistema}                   \\
\multicolumn{1}{l}{1) Preencher dados pessoais (nome, e-mail, senha)} & \multicolumn{4}{l}{2) Validar informações}                                     \\
\multicolumn{1}{l}{3) Confirmar cadastro}                             & \multicolumn{4}{l}{4) Registrar usuário no sistema}                            \\
\multicolumn{1}{l}{5) Enviar e-mail de confirmação}                   & \multicolumn{4}{l}{}                                                           \\
\cellcolor[HTML]{CCCCCC}Restrições/Validações                        & \multicolumn{4}{l}{\cellcolor[HTML]{FFFFFF}\begin{tabular}[c]{@{}l@{}}- E-mail deve ser único e válido\\ - Todos os campos obrigatórios devem ser preenchidos\end{tabular}} \\
\hline
\end{tabular}%
}
\end{table}

\subsubsection{UC02 - Gerenciar \textit{Ticket}}

A Tabela~\ref{tab:uc02_gerenciar_ticket} apresenta o caso de uso UC02, que descreve as ações relacionadas à criação, edição e visualização de \textit{tickets} por diferentes perfis de usuário no sistema, respeitando os níveis de acesso definidos.

\begin{table}[h]
\centering
\caption{Descrição do Caso de Uso UC02 - Gerenciar \textit{Ticket}}
\label{tab:uc02_gerenciar_ticket}
\resizebox{\textwidth}{!}{%
\begin{tabular}{cllll}
\hline
\rowcolor[HTML]{CCCCCC} 
Nome do caso de uso                                                                     & \multicolumn{4}{c}{\cellcolor[HTML]{CCCCCC}UC02 - Gerenciar \textit{Ticket}} \\ \hline
\cellcolor[HTML]{CCCCCC}Ator principal                                                  & \multicolumn{4}{l}{{\color[HTML]{333333} Cliente, Funcionário, Gerente}}      \\
\cellcolor[HTML]{CCCCCC}Atores secundários                                              & \multicolumn{4}{l}{-}                                                          \\
\cellcolor[HTML]{CCCCCC}Resumo                                                          & \multicolumn{4}{l}{\begin{tabular}[c]{@{}l@{}}Permite criar, alterar e visualizar \textit{tickets} no sistema com\\ diferentes níveis de acesso.\end{tabular}} \\ \hline
\cellcolor[HTML]{CCCCCC}Pré-condições                                                   & \multicolumn{4}{l}{Usuário deve estar autenticado no sistema.}                 \\
\cellcolor[HTML]{CCCCCC}Pós-condições                                                   & \multicolumn{4}{l}{Registro de \textit{ticket} atualizado no histórico do sistema.} \\
\rowcolor[HTML]{CCCCCC} 
\multicolumn{5}{c}{\cellcolor[HTML]{CCCCCC}Cenário Principal}                           \\
\rowcolor[HTML]{CCCCCC} 
Ações do ator                                                                           & \multicolumn{4}{c}{\cellcolor[HTML]{CCCCCC}Ações do sistema}                   \\
\multicolumn{1}{l}{1) Preencher dados do \textit{ticket} (título, descrição, categoria)}         & \multicolumn{4}{l}{2) Validar informações}                                     \\
\multicolumn{1}{l}{}                                                                    & \multicolumn{4}{l}{3) Registrar \textit{ticket} no sistema}                    \\
\multicolumn{1}{l}{4) Buscar \textit{ticket} existente}                                 & \multicolumn{4}{l}{5) Exibir detalhes completos}                               \\
\multicolumn{1}{l}{6) Modificar informações (status/descrição)}                         & \multicolumn{4}{l}{7) Atualizar registro}                                      \\
\cellcolor[HTML]{CCCCCC}Restrições/Validações                                           & \multicolumn{4}{l}{\cellcolor[HTML]{FFFFFF}\begin{tabular}[c]{@{}l@{}}- Histórico de alterações deve ser registrado\\ - Clientes só podem criar/visualizar próprios \textit{tickets}\\ - Gerentes têm acesso total a todos os \textit{tickets} do seu departamento\\ - Funcionários podem editar \textit{tickets} atribuídos a eles\end{tabular}} \\
\hline
\end{tabular}%
}
\end{table}


\newpage

\subsubsection{UC03 - Fazer Comentário}

A Tabela~\ref{tab:uc03_fazer_comentario} apresenta o caso de uso UC03, que descreve o processo de inclusão de comentários em \textit{tickets} já existentes. Essa funcionalidade é essencial para promover a comunicação entre usuários do sistema, permitindo que clientes e funcionários acompanhem o andamento e interajam nos chamados.

\begin{table}[h]
\centering
\caption{Descrição do Caso de Uso UC03 - Fazer Comentário}
\label{tab:uc03_fazer_comentario}
\resizebox{\textwidth}{!}{%
\begin{tabular}{cllll}
\hline
\rowcolor[HTML]{CCCCCC} 
Nome do caso de uso                                                                     & \multicolumn{4}{c}{\cellcolor[HTML]{CCCCCC}UC03 - Fazer Comentário}                         \\ \hline
\cellcolor[HTML]{CCCCCC}Ator principal                                                  & \multicolumn{4}{l}{{\color[HTML]{333333} Cliente, Funcionário}}                             \\
\cellcolor[HTML]{CCCCCC}Atores secundários                                              & \multicolumn{4}{l}{-}                                                                       \\
\cellcolor[HTML]{CCCCCC}Resumo                                                          & \multicolumn{4}{l}{\begin{tabular}[c]{@{}l@{}}Permite adicionar comentários aos \textit{tickets} para\\ comunicação entre as partes.\end{tabular}}                                     \\ \hline
\cellcolor[HTML]{CCCCCC}Pré-condições                                                   & \multicolumn{4}{l}{\begin{tabular}[c]{@{}l@{}}- \textit{Ticket} deve existir no sistema\\ - Usuário deve ter acesso ao \textit{ticket}\end{tabular}}                                            \\
\cellcolor[HTML]{CCCCCC}Pós-condições                                                   & \multicolumn{4}{l}{\begin{tabular}[c]{@{}l@{}}Comentário registrado com data/hora no \\ histórico do \textit{ticket}.\end{tabular}}                                                    \\
\rowcolor[HTML]{CCCCCC} 
\multicolumn{5}{c}{\cellcolor[HTML]{CCCCCC}Cenário Principal}                               \\
\rowcolor[HTML]{CCCCCC} 
Ações do Ator                                                                           & \multicolumn{4}{c}{\cellcolor[HTML]{CCCCCC}Ações do sistema}                                \\
\multicolumn{1}{l}{1) Preencher dados do comentário (título, descrição)}                & \multicolumn{4}{l}{\begin{tabular}[c]{@{}l@{}}2) Validar informações e se o \textit{ticket} do comentário\\ existe\end{tabular}}                                                       \\
\multicolumn{1}{l}{}                                                                    & \multicolumn{4}{l}{3) Registrar comentário no sistema}                                      \\
\multicolumn{1}{l}{}                                                                    & \multicolumn{4}{l}{4) Atualizar histórico do \textit{ticket}}                               \\
\cellcolor[HTML]{CCCCCC}Restrições/Validações                                           & \multicolumn{4}{l}{\cellcolor[HTML]{FFFFFF}\begin{tabular}[c]{@{}l@{}}- Cliente só comenta em próprios \textit{tickets}\\ - Funcionário só comenta em \textit{tickets} atribuídos\end{tabular}} \\
\hline
\end{tabular}%
}
\end{table}


\subsection{Diagrama de Sequência}

O diagrama de sequência é um artefato da UML que descreve a interação entre objetos ou componentes do sistema ao longo do tempo. Ele representa, em uma visão temporal, a troca de mensagens entre atores (Cliente, Controller, Service, Repositório) e evidencia a ordem e o fluxo das chamadas, desde a requisição até a resposta final. Esse diagrama é especialmente útil para:

\begin{itemize}
    \item Validar cenários de uso complexos, mostrando passo a passo como as operações são encadeadas.
    \item Identificar pontos de integração entre camadas (por exemplo, controller–service–repository).
    \item Documentar o comportamento dinâmico do sistema, facilitando a comunicação entre desenvolvedores e analistas.
\end{itemize}

A Figura \ref{fig:diagramadeSequenciaUsuario} demonstra o fluxo de criação do usuário, na qual, ao receber a requisição de criação de usuário, o \textit{Controller} delega a operação ao \textit{Service}. Este primeiro verifica se já existe um usuário com o mesmo e-mail ou CPF; em caso positivo, retorna um erro de duplicação (400 Bad Request). Se os dados forem válidos, o \textit{Service} codifica a senha (BCrypt), persiste a entidade no banco e dispara um e-mail de confirmação. Finalmente, retorna ao \textit{Controller} a resposta de sucesso (201 Created), que a envia de volta ao cliente.

\begin{figure}[h]
    \centering
    \includegraphics[width=\linewidth]{figuras/diagramas/sequencia/Diagrama de sequencia - CadastroUsuario.png}
    \caption{Diagrama de sequência do usuário}
    \label{fig:diagramadeSequenciaUsuario}
\end{figure}

\newpage
Já a Figura~\ref{fig:diagramadeSequenciaRegraPrioridade} ilustra o diagrama de sequência para o fluxo de criação de uma nova regra de prioridade. Esse diagrama evidencia a interação temporal entre os componentes do sistema: o {\tt Cliente}, o {\tt RegraPrioridadeController}, o serviço {\tt RegraPrioridadeServiceImpl}, os serviços de {\tt Categoria} e {\tt Departamento} e o {\tt RegraPrioridadeRepository}.

A Figura~\ref{fig:diagramadeSequenciaRegraPrioridade} ilustra, passo a passo, o processo de criação de uma nova regra de prioridade a partir de uma requisição feita por um cliente.

O fluxo se inicia quando o cliente envia os dados da nova regra ao \textit{Controller}. Em seguida, o \textit{Controller} repassa essas informações ao \textit{Service}, onde ocorre o processamento da lógica de negócio. Dentro do \textit{Service}, diversas etapas importantes são executadas:

\begin{itemize}
    \item Os dados recebidos são convertidos para o formato interno utilizado pelo sistema;
    \item O sistema verifica se a categoria e o departamento informados existem, consultando serviços especializados;
    \item Todos os dados são validados e preparados para persistência.
\end{itemize}

Após essas verificações, a nova regra é salva no banco de dados. Como medida de otimização, o sistema limpa o cache de regras, garantindo que as próximas consultas retornem informações atualizadas. Por fim, o sistema formata a resposta e a envia de volta ao cliente, confirmando a criação da regra com sucesso (código HTTP~201).


\begin{figure}[h]
    \centering
    \includegraphics[width=\linewidth]{figuras/diagramas/sequencia/Diagrama de sequencia - Regra Prioridade.png}
    \caption{Diagrama de sequência da Regra de Prioridade}
    \label{fig:diagramadeSequenciaRegraPrioridade}
\end{figure}

\newpage

\subsection{Diagrama de Classes de domínio}
A Figura~\ref{fig:diagramadeclasse} apresenta o diagrama de classes que representa a estrutura fundamental de um sistema de gerenciamento de \textit{tickets}, típico de aplicações como helpdesk ou suporte técnico. Abaixo, são detalhados os principais componentes e seus relacionamentos:

\subsubsection*{Principais Classes}

\begin{itemize}
    \item \textbf{\textit{Ticket}:} Classe central do sistema. Armazena informações essenciais como descrição, datas (criação, modificação, prazo), status atual e associações com outras entidades.
    
    \item \textbf{Usuario:} Representa os usuários cadastrados no sistema. Contém atributos como nome, e-mail, status e tipo de usuário, definido por um enumerador (\texttt{UsuarioRole}).
    
    \item \textbf{Departamento:} Modela os setores da organização. Inclui dados como nome, informações de contato e horário de funcionamento.
    
    \item \textbf{Comentario:} Permite que os usuários adicionem observações ou mensagens relacionadas aos \textit{tickets}.
    
    \item \textbf{TicketHistorico:} Armazena registros das alterações realizadas nos \textit{tickets}, funcionando como um log de mudanças.
    
    \item \textbf{ReparPrioridade:} Responsável pela gestão das prioridades dos \textit{tickets}, com métodos específicos para cálculo do nível de prioridade e do tempo restante.
    
    \item \textbf{Anexo:} Permite a associação de arquivos aos \textit{tickets} ou a outras entidades do sistema.
\end{itemize}

\subsubsection*{Tipos Enumerados (Enums)}

\begin{itemize}
    \item \textbf{StatusTicket:} Define os estados possíveis de um \textit{ticket}, como \texttt{Aberto}, \texttt{Em\_Andamento} e \texttt{Finalizado}.
    
    \item \textbf{UsuarioRole:} Enumeração que representa os papéis dos usuários no sistema, como \texttt{Administrador}, \texttt{Funcionário} e \texttt{Cliente}.
\end{itemize}

\begin{figure}[h]
    \centering
    \includegraphics[width=\linewidth]{figuras/diagramas/classe/Diagrama_de_classe.png}
    \caption{Diagrama de classe}
    \label{fig:diagramadeclasse}
\end{figure}

\newpage

\section{Definição de Tecnologias e Ferramentas}

Para o desenvolvimento do sistema, foram definidas as seguintes tecnologias, considerando critérios como robustez, compatibilidade com o escopo do projeto, comunidade ativa e facilidade de integração entre os componentes:

\begin{table}[h!]
    \centering
    \resizebox{\textwidth}{!}{%
        \begin{tabular}{|c|c|c|}
            \hline
            \textbf{Tecnologia} & \textbf{Aspecto} & \textbf{Descrição} \\ \hline
            Java & Linguagem & Desenvolvimento do Back-end (API REST) \\ \hline
            Spring Boot & Framework & Framework para aplicações Java \\ \hline
            RabbitMQ & Mensageria & Fila para comunicação assíncrona \\ \hline
            MySQL & SGBD & Banco de dados relacional \\ \hline
            IntelliJ IDEA & IDE & Desenvolvimento do Back-end \\ \hline
            Git & Versionamento de Código & Controle de versão local \\ \hline
            GitHub & Versionamento de Código & Armazenamento e colaboração em nuvem \\ \hline
            Postman & Testes de API & Testes e validação das requisições da API \\ \hline
        \end{tabular}%
    }
    \caption{Tabela consolidada de tecnologias, aspectos e descrições}
    \label{tab:consolidado_tecnologia}
\end{table}
               
\chapter*{CONCLUSÃO} 
\addcontentsline{toc}{chapter}{CONCLUSÃO}
\thispagestyle{plain}

Este trabalho teve como objetivo o desenvolvimento de uma API para gerenciamento de \textit{tickets} de atendimento, visando atender às necessidades de empresas que buscam melhorar seus processos internos de suporte e relacionamento com clientes. A solução proposta se destacou por sua flexibilidade, segurança e capacidade de personalização, oferecendo um sistema capaz de se adaptar às diferentes realidades organizacionais.

Ao longo do desenvolvimento, foram aplicados conceitos sólidos de engenharia de software, como a utilização de autenticação com JSON Web Token (JWT), arquitetura modular e boas práticas de segurança e escalabilidade. Além disso, o sistema foi estruturado para permitir integrações com outros serviços corporativos, oferecendo recursos como categorização de chamados, definição de níveis de prioridade, notificações automáticas e controle de fluxo de atendimento.

Os testes realizados demonstraram a eficácia da aplicação em cenários reais de uso, com funcionalidades completas de criação, atualização, listagem e exclusão de \textit{tickets}, bem como o registro de atividades relacionadas. A modelagem da base de dados, aliada ao uso de frameworks robustos, garantiu um desempenho satisfatório mesmo com grandes volumes de dados.

A Tabela~\ref{table:comparativo_tickets_final} apresenta um comparativo entre o sistema desenvolvido e três das principais ferramentas comerciais de gerenciamento de \textit{tickets} (Zendesk, Freshdesk e Zoho Desk), destacando-se aspectos como capacidade de personalização, integração com sistemas legados, controle de acesso e custo. Os dados demonstram que a solução criada oferece vantagens competitivas relevantes, especialmente para organizações que demandam maior flexibilidade e adaptação a contextos específicos.

\begin{table}[ht]
\centering
\resizebox{\textwidth}{!}{%
\begin{tabular}{|l|c|c|c|c|}
\hline
\textbf{Característica} & \textbf{Zendesk} & \textbf{Freshdesk} & \textbf{Zoho Desk} & \textbf{Sistema criado} \\ \hline
Interface amigável e intuitiva & Sim & Sim & Sim & Sim \\ \hline
Customização de fluxos e regras & Alta & Baixa & Média & Alta \\ \hline
Integração com sistemas legados & Limitada & Limitada & Limitada & Avançada \\ \hline
Automação de \textit{tickets} e respostas & Sim & Sim & Sim & Sim \\ \hline
Gerenciamento de SLA e métricas & Sim & Sim & Sim & Sim \\ \hline
Controle de acesso e permissões & Limitado & Limitado & Limitado & Avançado \\ \hline
Preço & Caro & Mais acessível & Mais acessível & Mais acessível \\ \hline
\end{tabular}
}
\caption{Comparativo de Características em Sistemas de Gerenciamento de \textit{Tickets} para o sistema criado}
\label{table:comparativo_tickets_final}
\end{table}

\newpage

Como trabalhos futuros, poderá ser realizada a implementação de novas funcionalidades no sistema, como um dashboard analítico com gráficos e indicadores que permitam a visualização em tempo real dos \textit{tickets} abertos, em andamento, resolvidos, tempo médio de atendimento, entre outros dados relevantes para a gestão de desempenho das equipes de suporte.
Também é proposta a integração com um chatbot e a criação de um módulo de feedback, permitindo que o cliente avalie o atendimento após o encerramento do \textit{ticket}, contribuindo para a melhoria contínua do serviço.

No entanto, este trabalho apresenta algumas limitações. Por se tratar de uma aplicação do tipo API, não foi possível realizar testes em um ambiente real de produção, uma vez que seria necessário disponibilizar a aplicação em um servidor para simular o uso contínuo. Além disso, a ausência de uma interface gráfica (front-end) dificultou a validação do sistema por usuários finais, restringindo os testes à utilização de ferramentas como o Postman ou o Swagger UI, o que limita a avaliação da usabilidade e da experiência do usuário. A criação de uma aplicação front-end integrada é, portanto, um passo essencial para uma avaliação mais completa da solução proposta.            
%---------------------------------------------------------------------
% BIBLIOGRAFIA
%		Edite o arquivo bibliografia, apenas acrescentando 
%		novas obras, não é necessário remover nada pois o 
%		latex insere somente as obras referenciadas. Ele também
%		ordena tudo!!! A configuração padrão deste modelo é a 
%		ordem alfabética (uma alternativa seria a ordem de 
%		citação no texto)
%       Para facilitar a confecção da bibliografia recomenda-se o 
% 		uso de aplicativos como Mendeley que gera arquivos .bib
%
%		IMPORTANTE: é necessário atualizar o documento todas 
%		as vezes que inserir um novo item na bibliografia. 
%		Procure o manual do seu editor para Saber como proceder.
%		Exemplo: TeXMaker 3.5.2
%		F11 para gerar o arquivo BBL
%       e para atualizar a bibliografia no documento:
%		menu <Editar>-opção <Atualizar Bibliografia> - atualize no arquivo
%		que contem a citação! 
%		pode ser necessário compilar/atualizar
%		mais de uma vez: NÃO SE ACANHE! SEJA INSISTENTE!
% 		
%--------------------------------------------------------------------
\bibliographystyle{abntex2-num}
%\bibliographystyle{plainnat}
\bibliography{postextuais/bibliografia}%inserindo a bibliografia
%%--------------------------------------------------------------------
%% ELEMENTOS POS-TEXTUAIS
%%		localize cada um dos arquivos na pasta postextuais
%%       e edite da forma que desejar! Caso não possua 
%%		apêndice ou anexos basta adicionar um sinal 
%%		de % na frente  do \include
%% 		
%%--------------------------------------------------------------------
%\printglossary                           %imprimindo o glossário
%\appendix

\chapter[Apêndice - IMAGENS E MODELOS DE CORES]{Imagens e modelos de cores} \label{sec:modeloscores}
%\setlength{\afterchapskip}{-\baselineskip}


 
%\anexos
\chapter{Documento de solicitação de ficha catalográfica}\label{anexo}
\setlength{\afterchapskip}{-\baselineskip}

%\chapter{Documento de solicitação de ficha catalográfica} \label{anexo}

\begin{figure}
\centering
\hspace{-2cm}
\includegraphics[scale=0.9]{figuras/solic_ficha_catalografica.pdf}
\end{figure}


Anexos devem conter informações e documentos que tenham sido confeccionados por terceiros e não pelo autor, ao contrário deste anexo, não é preciso inserir texto! Vale lembrar que não é estritamente necessário incluir anexos. 

Como exemplo este anexo traz o documento de solicitação de ficha catalográfica, a ser entregue na biblioteca, após a aprovação do trabalho em banca de defesa.
A versão final sem a ficha não será aceita, e a confecção da ficha antes da defesa também não, pois algumas informações só podem ser fornecidas após todas as correções.

\bigskip

\hspace{-2cm}

 
\end{document}